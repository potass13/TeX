%!TEX encoding = Shift-JIS
%Last update for this preamble : 2014-12-19
%pream01.tex -- jsarticle,,柱あり,listing 無

%
%usepackages
%
\usepackage{amsmath,amssymb}
\usepackage{bm}
\usepackage{ascmac}
\usepackage{braket}
\usepackage{comment}
%
%listing (comment out)
%
\begin{comment}
\usepackage{framed}
\usepackage{here}
\usepackage{listings, jlisting}
\renewcommand{\lstlistingname}{リスト}
\lstset{language=c,
  basicstyle=\ttfamily\scriptsize,
  commentstyle=\textit,
  classoffset=1,
  keywordstyle=\bfseries,
  frame=tRBl,
  framesep=5pt,
  showstringspaces=false,
  numbers=left,
  stepnumber=1,
  numberstyle=\tiny,
  tabsize=2
}
\end{comment}
%
%余白設定
%
\setlength{\textwidth}{\fullwidth}
\setlength{\textheight}{40\baselineskip}
\addtolength{\textheight}{\topskip}
\setlength{\voffset}{-0.2in}
\setlength{\topmargin}{0pt}
\setlength{\headheight}{0pt}
\setlength{\headsep}{10pt}
%
%柱の位置
%
\pagestyle{myheadings} % ページ番号&ヘッダ
%
%定理環境
%
\usepackage{theorem}
\theoremstyle{break}
\theorembodyfont{\normalfont}
\newtheorem{thm}{Thm}[section]
\newtheorem{defi}{Def}[section]
\newtheorem{lemma}{Lemma}[section]
\newtheorem{prop}{Prop}[section]
\newtheorem{proof}{Proof.}\renewcommand{\theproof}{}
%
\makeatletter 
%subsubのときの定理環境の修正
%「@」を含む場合は\makeatletterと\makeatotherで囲んだ範囲内で処理をする
\renewcommand{\subsubsection}{% \newcommand から \renewcommand に変更すること
  \@startsection{subsubsection}% 区切りコマンドの名前(section, subsection等)
    {3}% 深さ(sectionが1, subsectionは2等)
    {\z@}% 左のインデント量
    {\Cvs \@plus.5\Cdp \@minus.3\Cdp}% 前アキ 見出し上のスペース
    {.5\Cvs \@plus.4\Cdp}% 後アキ 見出し下のスペース 負にすると見出し後のスペース
    {\normalfont\normalsize\headfont\raggedright}% 見出しのフォント
}
\makeatother
%
%定義
%
\newcommand{\QED}{\rule[-2pt]{5pt}{10pt}} % 証明終了
%
\renewcommand{\figurename}{Fig.} % 図 -> Fig.
\renewcommand{\tablename}{Tab.} % 表 -> Tab.
%
\def\defarrow{\overset{\mathrm{def}}{\Longleftrightarrow}} % <= def =>
%
\def\sgn{\mathrm{sgn}\,} % sgn
\def\per{\mathrm{per}\,} % per
\newcommand{\diver}{\mathrm{div}\,} % div
\newcommand{\grad}{\mathrm{grad}\,} % grad
\newcommand{\rot}{\mathrm{rot}\,} % rot
%
\def\MARU#1{{\rm\ooalign{\hfil\lower.168ex\hbox{#1}\hfil \crcr\mathhexbox20D}}} % 丸文字
%
\newcommand{\simgt}{\lower.5ex\hbox{$\; \buildrel > \over \sim \;$}} % 近似不等式 >=
\newcommand{\simlt}{\lower.5ex\hbox{$\; \buildrel < \over \sim \;$}} % 近似不等式 <=
